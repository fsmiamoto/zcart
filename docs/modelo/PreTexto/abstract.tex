%%%% ABSTRACT
%%
%% Versão do resumo para idioma de divulgação internacional.

\begin{abstractutfpr}%% Ambiente abstractutfpr
Os recentes avanços em inteligência artificial já impactaram diversos aspectos
da vida moderna mas ainda não atingiram um aspecto importante da
experiência dos consumidores: compras em lojas físicas. Gigantes da
tecnologia como a Amazon lançaram recentemente os chamados
\textit{carrinhos inteligentes} (smart carts) em suas lojas físicas,
proporcionando aos consumidores uma melhor experiência de compra, com mais
informações sobre os produtos e um processo de pagamento rápido e prático.
Neste sentido, o presente trabalho analisa o contexto atual do mercado e
descreve o desenvolvimento de um protótipo que entrega funcionalidades
similares aos produtos disponíveis no mercado utilizando as mesmas bases
tecnológicas de visão computacional e sensores. Um modelo de
aprendizado profundo (Deep Learning) foi desenvolvido para a detecção de
produtos e implantado em um Single Board Computer capaz de executar
    inferências em aproximadamente 4 \siglaIt{Quadros Por Segundo}{QPS} com uma
precisão média acima dos 80\%. Finalmente, o trabalho discute os desafios e
restrições práticas do desenvolvimento do protótipo e prepara o caminho
para trabalhos futuros que podem levar o desenvolvimento até um produto
comercial.
\end{abstractutfpr}
